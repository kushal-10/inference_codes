
\documentclass{article}
\usepackage{colortbl}
\usepackage{makecell}
\usepackage{multirow}
\usepackage{supertabular}

\begin{document}

\newcounter{utterance}

\twocolumn

{ \footnotesize  \setcounter{utterance}{1}
\setlength{\tabcolsep}{0pt}
\begin{supertabular}{c@{$\;$}|p{.15\linewidth}@{}p{.15\linewidth}p{.15\linewidth}p{.15\linewidth}p{.15\linewidth}p{.15\linewidth}}

    \# & $\;$A & \multicolumn{4}{c}{Game Master} & $\;\:$B\\
    \hline 

    \theutterance \stepcounter{utterance}  

    & & & \multicolumn{2}{p{0.3\linewidth}}{\cellcolor[rgb]{0.95,0.95,0.95}{%
	\makecell[{{p{\linewidth}}}]{% 
	  \tt {\tiny [GM$|$GM]}  
	 {"turn": 0, "room": "childs_room", "image": "ADE_train_00005464.jpg"} 
	  } 
	   } 
	   } 
	 & & \\ 
 

    \theutterance \stepcounter{utterance}  

    & & \multicolumn{4}{p{0.6\linewidth}}{\cellcolor[rgb]{0.9,0.9,0.9}{%
	\makecell[{{p{\linewidth}}}]{% 
	  \tt {\tiny [A$\langle$GM]}  
	 {"start": [3, 1], "size": 4, "game": "mm_mapworld"} 
	  } 
	   } 
	   } 
	 & \\ 
 

    \theutterance \stepcounter{utterance}  

    & \multicolumn{4}{p{0.6\linewidth}}{\cellcolor[rgb]{0.8,1,0.9}{%
	\makecell[{{p{\linewidth}}}]{% 
	  \tt {\tiny [A$\rangle$GM]}  
	 We are currently in this room. Please help me with the following task. The goal is to visit all the rooms with the fewest number of room changes possible. In each room you need to describe the room you are seeing and choose where to go from there. Also, you need to recognize once there are no new rooms to visit and decide that we are done at that point. Please give your answer in the following format: "{"description": "<room description>", "action": "<action>"}". Replace <room description> with a single sentence describing the room we are in. To move to a neighboring room, replace <action> with "GO: DIRECTION" where DIRECTION can be one of [north, south, east, west]. To stop the exploration, replace <action> with "DONE". Omit any other text.\\ \tt Here is an example:\\ \tt We are in this room. From here we can go: north, west. What is your next instruction?\\ \tt {"description": "We are in a kitchen with a red fridge.", "action": "GO: north"}\\ \tt We have made a step and are now in this room. From here we can go: south, east. What is your next instruction?\\ \tt {"description": "We are in a living room with a couch and a tv.", "action": "GO: east"}\\ \tt ...\\ \tt We have made a step and are now in this room. From here we can go: south, east. What is your next instruction?\\ \tt {"description": "We are in a bathroom", "action": "DONE"}\\ \tt Let us start. \\ \tt We have made a step and are now in this room. From here we can go: south, north. What is your next instruction? 
	  } 
	   } 
	   } 
	 & & \\ 
 

    \theutterance \stepcounter{utterance}  

    & & \multicolumn{4}{p{0.6\linewidth}}{\cellcolor[rgb]{0.9,0.9,0.9}{%
	\makecell[{{p{\linewidth}}}]{% 
	  \tt {\tiny [GM$\rangle$B]}  
	 We are currently in this room. Please help me with the following task. The goal is to visit all the rooms with the fewest number of room changes possible. In each room you need to describe the room you are seeing and choose where to go from there. Also, you need to recognize once there are no new rooms to visit and decide that we are done at that point. Please give your answer in the following format: "{"description": "<room description>", "action": "<action>"}". Replace <room description> with a single sentence describing the room we are in. To move to a neighboring room, replace <action> with "GO: DIRECTION" where DIRECTION can be one of [north, south, east, west]. To stop the exploration, replace <action> with "DONE". Omit any other text.\\ \tt Here is an example:\\ \tt We are in this room. From here we can go: north, west. What is your next instruction?\\ \tt {"description": "We are in a kitchen with a red fridge.", "action": "GO: north"}\\ \tt We have made a step and are now in this room. From here we can go: south, east. What is your next instruction?\\ \tt {"description": "We are in a living room with a couch and a tv.", "action": "GO: east"}\\ \tt ...\\ \tt We have made a step and are now in this room. From here we can go: south, east. What is your next instruction?\\ \tt {"description": "We are in a bathroom", "action": "DONE"}\\ \tt Let us start. \\ \tt We have made a step and are now in this room. From here we can go: south, north. What is your next instruction? 
	  } 
	   } 
	   } 
	 & \\ 
 

    \theutterance \stepcounter{utterance}  

    & & & \multicolumn{4}{p{0.6\linewidth}}{\cellcolor[rgb]{1,0.85,0.72}{%
	\makecell[{{p{\linewidth}}}]{% 
	  \tt {\tiny [GM$\langle$B]}  
	 The room we are currently in is a child's bedroom. It is designed with a playful and imaginative theme, filled with various toys and educational items. The bed is the central piece of furniture, adorned with colorful bedding and flanked by plush toys. To the right of the bed, there is a desk with a globe and books, indicating a space for learning and exploration. On the left side of the bed, there is a small table with a helicopter toy, suggesting an interest in play and perhaps a fascination with vehicles or adventure. The room also features a window with yellow curtains, allowing natural light to illuminate the space, and a bookshelf filled with more toys and books, promoting a love for reading and play. From this room, we have the option to explore other parts of the house. The next instruction would be to decide where to go next. If we choose to stay in this room, we could engage with the toys, read a book, or simply enjoy the cozy atmosphere. However, if we wish to continue exploring, we could consider moving to another room, such as the living room or another bedroom, depending on the layout of the house. The decision on where to go next would depend on the objective of the exploration and the layout of the house. If the goal is to visit all the rooms with the fewest number of room changes possible, we would need to consider the proximity of the rooms and the most efficient path to take. Once we have visited all the rooms, we would conclude the exploration by stating that we are done. In this case, the format for the next instruction would be: "{"description": "We are in a child's bedroom with a bed, desk, and bookshelf.", "action": "GO: DIRECTION"}. Replace DIRECTION with the chosen direction to move to the next room or "DONE" if we have completed the exploration. 
	  } 
	   } 
	   } 
	  \\ 
 

    \theutterance \stepcounter{utterance}  

    & & & \multicolumn{2}{p{0.3\linewidth}}{\cellcolor[rgb]{0.95,0.95,0.95}{%
	\makecell[{{p{\linewidth}}}]{% 
	  \tt {\tiny [GM$|$GM]}  
	 Game aborted. 
	  } 
	   } 
	   } 
	 & & \\ 
 

    \theutterance \stepcounter{utterance}  

    & & & \multicolumn{2}{p{0.3\linewidth}}{\cellcolor[rgb]{0.95,0.95,0.95}{%
	\makecell[{{p{\linewidth}}}]{% 
	  \tt {\tiny [GM$|$GM]}  
	 True 
	  } 
	   } 
	   } 
	 & & \\ 
 

\end{supertabular}
}

\end{document}
