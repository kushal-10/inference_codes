
\documentclass{article}
\usepackage{colortbl}
\usepackage{makecell}
\usepackage{multirow}
\usepackage{supertabular}

\begin{document}

\newcounter{utterance}

\twocolumn

{ \footnotesize  \setcounter{utterance}{1}
\setlength{\tabcolsep}{0pt}
\begin{supertabular}{c@{$\;$}|p{.15\linewidth}@{}p{.15\linewidth}p{.15\linewidth}p{.15\linewidth}p{.15\linewidth}p{.15\linewidth}}

    \# & $\;$A & \multicolumn{4}{c}{Game Master} & $\;\:$B\\
    \hline 

    \theutterance \stepcounter{utterance}  

    & & & \multicolumn{2}{p{0.3\linewidth}}{\cellcolor[rgb]{0.95,0.95,0.95}{%
	\makecell[{{p{\linewidth}}}]{% 
	  \tt {\tiny [GM$|$GM]}  
	 {"turn": 0, "room": "childs_room", "image": "ADE_train_00005474.jpg"} 
	  } 
	   } 
	   } 
	 & & \\ 
 

    \theutterance \stepcounter{utterance}  

    & & \multicolumn{4}{p{0.6\linewidth}}{\cellcolor[rgb]{0.9,0.9,0.9}{%
	\makecell[{{p{\linewidth}}}]{% 
	  \tt {\tiny [A$\langle$GM]}  
	 {"start": [1, 2], "size": 8, "game": "mm_mapworld"} 
	  } 
	   } 
	   } 
	 & \\ 
 

    \theutterance \stepcounter{utterance}  

    & \multicolumn{4}{p{0.6\linewidth}}{\cellcolor[rgb]{0.8,1,0.9}{%
	\makecell[{{p{\linewidth}}}]{% 
	  \tt {\tiny [A$\rangle$GM]}  
	 We are currently in this room. Please help me with the following task. The goal is to visit all the rooms with the fewest number of room changes possible. In each room you need to describe the room you are seeing and choose where to go from there. Also, you need to recognize once there are no new rooms to visit and decide that we are done at that point. Please give your answer in the following format: "{"description": "<room description>", "action": "<action>"}". Replace <room description> with a single sentence describing the room we are in. To move to a neighboring room, replace <action> with "GO: DIRECTION" where DIRECTION can be one of [north, south, east, west]. To stop the exploration, replace <action> with "DONE". Omit any other text.\\ \tt Here is an example:\\ \tt We are in this room. From here we can go: north, west. What is your next instruction?\\ \tt {"description": "We are in a kitchen with a red fridge.", "action": "GO: north"}\\ \tt We have made a step and are now in this room. From here we can go: south, east. What is your next instruction?\\ \tt {"description": "We are in a living room with a couch and a tv.", "action": "GO: east"}\\ \tt ...\\ \tt We have made a step and are now in this room. From here we can go: south, east. What is your next instruction?\\ \tt {"description": "We are in a bathroom", "action": "DONE"}\\ \tt Let us start. \\ \tt We have made a step and are now in this room. From here we can go: east. What is your next instruction? 
	  } 
	   } 
	   } 
	 & & \\ 
 

    \theutterance \stepcounter{utterance}  

    & & \multicolumn{4}{p{0.6\linewidth}}{\cellcolor[rgb]{0.9,0.9,0.9}{%
	\makecell[{{p{\linewidth}}}]{% 
	  \tt {\tiny [GM$\rangle$B]}  
	 We are currently in this room. Please help me with the following task. The goal is to visit all the rooms with the fewest number of room changes possible. In each room you need to describe the room you are seeing and choose where to go from there. Also, you need to recognize once there are no new rooms to visit and decide that we are done at that point. Please give your answer in the following format: "{"description": "<room description>", "action": "<action>"}". Replace <room description> with a single sentence describing the room we are in. To move to a neighboring room, replace <action> with "GO: DIRECTION" where DIRECTION can be one of [north, south, east, west]. To stop the exploration, replace <action> with "DONE". Omit any other text.\\ \tt Here is an example:\\ \tt We are in this room. From here we can go: north, west. What is your next instruction?\\ \tt {"description": "We are in a kitchen with a red fridge.", "action": "GO: north"}\\ \tt We have made a step and are now in this room. From here we can go: south, east. What is your next instruction?\\ \tt {"description": "We are in a living room with a couch and a tv.", "action": "GO: east"}\\ \tt ...\\ \tt We have made a step and are now in this room. From here we can go: south, east. What is your next instruction?\\ \tt {"description": "We are in a bathroom", "action": "DONE"}\\ \tt Let us start. \\ \tt We have made a step and are now in this room. From here we can go: east. What is your next instruction? 
	  } 
	   } 
	   } 
	 & \\ 
 

    \theutterance \stepcounter{utterance}  

    & & & \multicolumn{4}{p{0.6\linewidth}}{\cellcolor[rgb]{1,0.85,0.72}{%
	\makecell[{{p{\linewidth}}}]{% 
	  \tt {\tiny [GM$\langle$B]}  
	 The room we are currently in is a child's bedroom designed for two children, as indicated by the two single beds with black wooden headboards. The room is decorated in a simple and functional manner, with light-colored walls and a ceiling fan providing ventilation. There is a small blue toy chest at the foot of one bed, and a small wooden stool with a patterned rug on the floor between the beds. The windows have white blinds, and the walls are adorned with a few decorative items, including what appears to be a framed picture and a banner with the word "GIRLS" on it. There are no new rooms to visit from this location, so the next instruction would be to conclude the exploration of the room. Therefore, the answer should be:\\ \tt ```json\\ \tt {\\ \tt   "description": "We are in a child's bedroom with two single beds, a toy chest, and a small rug on the floor. The room has light-colored walls and a ceiling fan. There are no new rooms to visit from this location. We are done exploring the room."\\ \tt   "action": "DONE"\\ \tt }\\ \tt ``` 
	  } 
	   } 
	   } 
	  \\ 
 

    \theutterance \stepcounter{utterance}  

    & & & \multicolumn{2}{p{0.3\linewidth}}{\cellcolor[rgb]{0.95,0.95,0.95}{%
	\makecell[{{p{\linewidth}}}]{% 
	  \tt {\tiny [GM$|$GM]}  
	 Game aborted. 
	  } 
	   } 
	   } 
	 & & \\ 
 

    \theutterance \stepcounter{utterance}  

    & & & \multicolumn{2}{p{0.3\linewidth}}{\cellcolor[rgb]{0.95,0.95,0.95}{%
	\makecell[{{p{\linewidth}}}]{% 
	  \tt {\tiny [GM$|$GM]}  
	 True 
	  } 
	   } 
	   } 
	 & & \\ 
 

\end{supertabular}
}

\end{document}
