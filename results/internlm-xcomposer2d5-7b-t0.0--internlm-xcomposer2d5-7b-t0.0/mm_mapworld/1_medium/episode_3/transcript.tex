
\documentclass{article}
\usepackage{colortbl}
\usepackage{makecell}
\usepackage{multirow}
\usepackage{supertabular}

\begin{document}

\newcounter{utterance}

\twocolumn

{ \footnotesize  \setcounter{utterance}{1}
\setlength{\tabcolsep}{0pt}
\begin{supertabular}{c@{$\;$}|p{.15\linewidth}@{}p{.15\linewidth}p{.15\linewidth}p{.15\linewidth}p{.15\linewidth}p{.15\linewidth}}

    \# & $\;$A & \multicolumn{4}{c}{Game Master} & $\;\:$B\\
    \hline 

    \theutterance \stepcounter{utterance}  

    & & & \multicolumn{2}{p{0.3\linewidth}}{\cellcolor[rgb]{0.95,0.95,0.95}{%
	\makecell[{{p{\linewidth}}}]{% 
	  \tt {\tiny [GM$|$GM]}  
	 {"turn": 0, "room": "youth_hostel", "image": "ADE_train_00020145.jpg"} 
	  } 
	   } 
	   } 
	 & & \\ 
 

    \theutterance \stepcounter{utterance}  

    & & \multicolumn{4}{p{0.6\linewidth}}{\cellcolor[rgb]{0.9,0.9,0.9}{%
	\makecell[{{p{\linewidth}}}]{% 
	  \tt {\tiny [A$\langle$GM]}  
	 {"start": [1, 0], "size": 6, "game": "mm_mapworld"} 
	  } 
	   } 
	   } 
	 & \\ 
 

    \theutterance \stepcounter{utterance}  

    & \multicolumn{4}{p{0.6\linewidth}}{\cellcolor[rgb]{0.8,1,0.9}{%
	\makecell[{{p{\linewidth}}}]{% 
	  \tt {\tiny [A$\rangle$GM]}  
	 We are currently in this room. Please help me with the following task. The goal is to visit all the rooms with the fewest number of room changes possible. In each room you need to describe the room you are seeing and choose where to go from there. Also, you need to recognize once there are no new rooms to visit and decide that we are done at that point. Please give your answer in the following format: "{"description": "<room description>", "action": "<action>"}". Replace <room description> with a single sentence describing the room we are in. To move to a neighboring room, replace <action> with "GO: DIRECTION" where DIRECTION can be one of [north, south, east, west]. To stop the exploration, replace <action> with "DONE". Omit any other text.\\ \tt Here is an example:\\ \tt We are in this room. From here we can go: north, west. What is your next instruction?\\ \tt {"description": "We are in a kitchen with a red fridge.", "action": "GO: north"}\\ \tt We have made a step and are now in this room. From here we can go: south, east. What is your next instruction?\\ \tt {"description": "We are in a living room with a couch and a tv.", "action": "GO: east"}\\ \tt ...\\ \tt We have made a step and are now in this room. From here we can go: south, east. What is your next instruction?\\ \tt {"description": "We are in a bathroom", "action": "DONE"}\\ \tt Let us start. \\ \tt We have made a step and are now in this room. From here we can go: east. What is your next instruction? 
	  } 
	   } 
	   } 
	 & & \\ 
 

    \theutterance \stepcounter{utterance}  

    & & \multicolumn{4}{p{0.6\linewidth}}{\cellcolor[rgb]{0.9,0.9,0.9}{%
	\makecell[{{p{\linewidth}}}]{% 
	  \tt {\tiny [GM$\rangle$B]}  
	 We are currently in this room. Please help me with the following task. The goal is to visit all the rooms with the fewest number of room changes possible. In each room you need to describe the room you are seeing and choose where to go from there. Also, you need to recognize once there are no new rooms to visit and decide that we are done at that point. Please give your answer in the following format: "{"description": "<room description>", "action": "<action>"}". Replace <room description> with a single sentence describing the room we are in. To move to a neighboring room, replace <action> with "GO: DIRECTION" where DIRECTION can be one of [north, south, east, west]. To stop the exploration, replace <action> with "DONE". Omit any other text.\\ \tt Here is an example:\\ \tt We are in this room. From here we can go: north, west. What is your next instruction?\\ \tt {"description": "We are in a kitchen with a red fridge.", "action": "GO: north"}\\ \tt We have made a step and are now in this room. From here we can go: south, east. What is your next instruction?\\ \tt {"description": "We are in a living room with a couch and a tv.", "action": "GO: east"}\\ \tt ...\\ \tt We have made a step and are now in this room. From here we can go: south, east. What is your next instruction?\\ \tt {"description": "We are in a bathroom", "action": "DONE"}\\ \tt Let us start. \\ \tt We have made a step and are now in this room. From here we can go: east. What is your next instruction? 
	  } 
	   } 
	   } 
	 & \\ 
 

    \theutterance \stepcounter{utterance}  

    & & & \multicolumn{4}{p{0.6\linewidth}}{\cellcolor[rgb]{1,0.85,0.72}{%
	\makecell[{{p{\linewidth}}}]{% 
	  \tt {\tiny [GM$\langle$B]}  
	 The room we are currently in appears to be a simple, utilitarian sleeping area, possibly in a hostel or budget accommodation. It features two single beds with basic bedding, a ladder leading to a top bunk, and a large window that allows natural light to fill the room. There are no decorations or personal items visible, suggesting that the room is meant for temporary stays rather than long-term residence. The presence of luggage on the floor indicates that guests have recently arrived or are preparing to depart. To proceed with the task, we need to identify the neighboring rooms and determine the best path to visit them with the fewest changes. However, without additional context or information about the layout of the entire building, it is challenging to provide a definitive next instruction. If we assume that the room we are in is part of a larger complex with adjacent rooms, we could explore possible directions such as north, south, east, or west. However, without further details, it is not possible to give a precise answer. Therefore, I would recommend gathering more information about the surrounding rooms and their accessibility before making a decision on the next move. 
	  } 
	   } 
	   } 
	  \\ 
 

    \theutterance \stepcounter{utterance}  

    & & & \multicolumn{2}{p{0.3\linewidth}}{\cellcolor[rgb]{0.95,0.95,0.95}{%
	\makecell[{{p{\linewidth}}}]{% 
	  \tt {\tiny [GM$|$GM]}  
	 Game aborted. 
	  } 
	   } 
	   } 
	 & & \\ 
 

    \theutterance \stepcounter{utterance}  

    & & & \multicolumn{2}{p{0.3\linewidth}}{\cellcolor[rgb]{0.95,0.95,0.95}{%
	\makecell[{{p{\linewidth}}}]{% 
	  \tt {\tiny [GM$|$GM]}  
	 True 
	  } 
	   } 
	   } 
	 & & \\ 
 

\end{supertabular}
}

\end{document}
